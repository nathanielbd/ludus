% I'm writing this intro to be read following the abstract.

Deck-building games are a genre of card games where each player
constructs a deck by combining cards from a heterogeneous pool, often
called a \newterm{set,} according to some \newterm{deck-building
  rules.} Players then \newterm{pilot} their decks in games against
other players' decks by strategically selecting which cards to play as
each unique game unfolds. Because it takes place prior to and effects
the outcomes of multiple individual games, the deck-building process
is often called the \newterm{metagame.}  Metagames are said to be
\newterm{healthy} when there are many diverse \enquote{archetypes}, or
categories of decks, available for players to choose between.
% FIXME: this clause is imprecise, & invites criticism re: "how do you
% measure how fun a game is"
% and when most games between these decks are fun.
% consider replacing with "enabling a wider array of strategic
% deck-building decisions and more varied, and therefore more
% interesting, individual games."
Deck-building games pose a challenge for game designers, since a
relatively small number of cards can be combined into a large number
of possible decks, which in turn can be paired into an even larger
number of match-ups. Human playtesting has repeatedly failed to
identify card designs or sets that lead to unhealthy metagames.
% TODO: do we need to cite sources?
% possible examples:
% - hearthstone's "miracle rogue" and "grim patron" metagames
% - magic: the gathering's "combo winter," "affinity," "caw-blade" and "oko" metagames
% - yu-gi-oh, i assume
% - to a lesser extent, "good boy" in storybook brawl
Even when it does successfully identify and fix unhealthy metagames,
human playtesting is often prohibitively expensive, requiring
significant labor and time from a number of skilled players. In this
paper, we attempt to reduce this cost by automating the balancing
process.
% this is where I cut off because I'm going to a movie.
