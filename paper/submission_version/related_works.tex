As the auto battler genre is still new, we were only able to find one other work
that specifically addresses the use of AI in their design.  \citeauthor{tencent_autobattle_lineup}
\shortcite{tencent_autobattle_lineup} focused on autonomously identifying the
best decks that players can construct given the available cards, game rules, and
decks that the game's designers expect to be played the most\footnote{Due to
Tencent's own game being the specific application, they use different terminology.
A `piece' is a `card,' and a `lineup' is a `deck'.}.  In a two-step process,
their method first simulates play between randomly generated decks and the designers'
expected decks to create a training dataset for a neural network---the model
learns a mapping between a given deck and its estimated win rate against the
designers' expected decks.  In the second step, a genetic algorithm searches for
decks that optimize the learned win rate function under various constraints that
define deck construction rules throughout the auto battler game.  Constraints
account for drafting additional cards between rounds of play and rules that
increase the strength of cards based on draft choices, which are not elements of
our simplified auto battler game (see Section~\ref{sec:ab-game-def} for a description).

The optimized decks are shared with the game designers who may use the
analysis to evaluate whether they believe the cards are balanced and, if not, which
cards need revision.  Our method instead revises the cards directly, optimizing
the viability of the set of possible decks that players can construct.  This alters
the initial efforts of the game designers to focus on the rules and some card templates
without committing to specific grounded card instances.  Because the templates are
a lifted representation of the actual game's cards, it is more difficult for designers
to identify which decks are expected to be common in the metagame.  However, after
our approach provides suggestions for grounded card instances, the game designers
can apply \citeauthor{tencent_autobattle_lineup}'s methods for a deeper analysis
of the suggested card designs.  If the designers have some expected decks based
on the templates alone, then it should be possible to combine our methods to
present alternative optimization criteria.  The information available to the game
designers would enable them to explore not just which decks are considered the
best for given card designs, but also how the best decks change for various card
revisions.

Aside from deck-building games in the auto battler genre, many games that support competitive
play over the internet collect data about what and how people are playing.  Game
designers can take advantage of various data science approaches to find trends in
the logged data that inform them about how to balance the game \cite{nce-gameAIpro2}.
In the virtual collectible card game Hearthstone, clusters of similar deck
compositions implicitly describe archetypes that are currently popular in the
metagame.  Designers can use this information to decide if some archetypes are too
common (implying they might be too powerful) and respond through updates to card
descriptions, releasing new cards that counter the overused archetypes, or
releasing new cards that support the underused archetypes \cite{blizzard-gamebalancetalk-keg2019}.
This effectively organizes human-operated playtesting at macro-scale where designers
can iterate on their game between update releases.

Because human-operated playtesting is expensive resource-wise and rarely
exhaustive enough to identify all points of concern, automating the playtesting
process can speed up the number of games played and discover edge cases humans
might not consider.  For deck-building games where the players have
agency and outcomes are nondeterministic (shuffled decks, random outcomes of effects,
etc.), the space of possible deck combinations and game states can be immeasurable.
Following in the footsteps of DeepMind's work on AlphaGo \cite{alphago}, Stadia
trained a deep learning model through many games of self-play in order to develop
a function that could evaluate the quality of a game state for a custom deck-building
game they created called Chimera \cite{chimera-mlagent}.  The authors provide
little information to replicate their work---it is not mentioned how their automated
game-playing agent uses this learned model to make decisions during gameplay
(AlphaGo uses Monte Carlo Tree Search, but self-play can also imply reinforcement
learning of a policy), nor is it clear whether the decks used in self-play were
the same or different throughout training.  However, they illustrated how their
automated game-playing agent could compete against itself multiple times with
designer-constructed decks in order to compare the overall decks' performance,
accounting for the nondeterministic outcomes through many games as statistical
samples.  A game designer may manually inspect the outcomes of all the matches
to determine whether any revisions to the cards are necessary.

Auto battler games lack this degree of uncertainty once the two decks begin combat.
Thus, \citeauthor{tencent_autobattle_lineup}'s \shortcite{tencent_autobattle_lineup}
and our methods focus automated playtesting towards a breadth of decks competing
against each other rather than depth exploring all possible games between a few
decks.  As there are still too many decks to consider for feasible automated
playtesting, we present an approximation algorithm that estimates a deck's overall
performance through random match-ups (Sections~\ref{sec:tourney} and~\ref{sec:metrics}) while
\citeauthor{tencent_autobattle_lineup}'s neural network estimates a deck's overall
performance from the offline match-ups against designer-crafted decks that generate
their training data.  Allowing designers to consider play against select decks
rather than the entire field can more efficiently balance larger metagames.
In addition to requiring less computation online for automated playtesting, this
might approximate human preferences with greater accuracy because players
tend to avoid building decks that appear weaker; our approximation algorithm
considers all decks, including these weaker ones, during automated playtesting.
%Integrating these methods into our optimization framework could allow designers
%of larger sets to more efficiently balance large metagames by considering only
%strong decks rather than evaluating the entire field. In addition to
%requiring less computation, this would better approximate human play
%patterns, since human players tend to avoid building weak decks, and
%therefore evaluating a candidate deck's performance against weak decks
%matters less.
