% I'm writing this intro to be read following the abstract.

% TODO consider subsection header placement and titles

Deck-building games are a genre of card games where each player
constructs a deck by combining cards from a heterogeneous pool, often
called a \newterm{set,} according to some \newterm{deck-building
rules.} Players then \newterm{pilot} their decks in games against
other players' decks by strategically selecting which cards to play as
each unique game unfolds. Because it takes place prior to and effects
the outcomes of multiple individual games, the deck-building process
is often called the \newterm{metagame.}  Metagames are said to be
\newterm{healthy} when there are many diverse \newterm{archetypes}, or
categories of decks, available for players to choose between.
% FIXME: this clause is imprecise, & invites criticism re: "how do you
% measure how fun a game is"
% and when most games between these decks are fun.
% consider replacing with "enabling a wider array of strategic
% deck-building decisions and more varied, and therefore more
% interesting, individual games."

% TODO: why are healthy metagames important?
Healthy metagames are desirable (or conversely, unhealthy metagames
are undesirable) in part because they alienate players, leading to
reduced engagement and revenues. Quantifying these risks is difficult,
as most game studios do not publish marketing data in enough detail
for the public to classify particular products and their metagames as
successes or flops. However, discussions by enfranchised players on
online forums can give us a qualitative picture of the associated
risks, and balancing actions taken by game studios can provide
insights into how seriously they view the danger posed by an unhealthy
metagame. For example, \textit{Magic: the Gathering}'s set
\textit{Throne of Eldraine} saw criticism online for being
significantly more powerful than the other cards in Standard,
\textit{Magic}'s most popular competitive format, and on at least one
occasion led to tournaments being cancelled due to lack of interest
\cite{oko-meta-drama}. Wizards of the Coast responded by banning six
cards from \textit{Eldraine} from Standard play across three separate
``banwaves'' \cite{mtg-banlist, mtg-bnr-nov-2019, mtg-bnr-jun-2020,
mtg-bnr-aug-2020}; these correspond with dramatic drops in banned
cards' prices on the secondary market
\cite{tcgplayer-bans-financial-impact}. Even after such bans,
high-profile professional \textit{Magic} players have taken to Twitter
to complain about \textit{Eldraine}'s affect on the Standard metagame
\cite{lsv-eldraine-complaints}.

% TODO: should i write about any other games? should i write about the
% companion rules change?

Despite its importance, metagame balance in deck-building games poses
a challenge for game designers, since a relatively small number of
cards can be combined into a large number of possible decks, which in
turn can be paired into an even larger number of
match-ups. \textit{Magic: the Gathering} relies on human playtesting
in advance of releasing a set \cite{designing-hod-ffl}, and digital
card games like \textit{Hearthstone} also collect and analyze data
after release from consumer play,
\cite{blizzard-gamebalancetalk-keg2019}, but we are not aware of any
studios which use automated playtesting to reduce human involvement.
Advance human playtesting has repeatedly failed to identify card
designs or sets that lead to unhealthy metagames, and even when it
does successfully identify and fix unhealthy metagames, human
playtesting is often prohibitively expensive, requiring significant
labor and time from a number of skilled players. Analyzing consumer
play data requires significant number of consumers to play in
unhealthy metagames in order to identify problematic designs.
% TODO: do we need to cite sources?
% possible examples:
% - hearthstone's "miracle rogue" and "grim patron" metagames
% - magic: the gathering's "combo winter," "affinity," "caw-blade" and "oko" metagames
% - yu-gi-oh, i assume
% - to a lesser extent, "good boy" in storybook brawl
In this paper, we attempt to reduce the cost of playtesting without
exposing consumers to unhealthy metagames by automating the balancing
process.

Our research focuses on auto battler games, a genre pioneered in 2019
by \textit{Dota Auto Chess} \cite{autochess}, with other notable
examples being \textit{Hearthstone Battlegrounds}
\cite{hearthstone-battlegrounds} and \textit{Storybook Brawl}
\cite{storybook-brawl}.
% TODO: cite these?
In an auto battler, each game (sometimes called a \newterm{round} or
\newterm{battle}) plays out automatically, with no input from the
players. Players' only strategic decisions are made during
deck-building, and the battles serve as automated evaluations of the
players' deck-building choices. Typically, eight players enter a
\newterm{queue}, where they purchase cards from a shared pool,
iteratively improving their decks in a deck-building phase between
each battle. In our research, we focus on balancing the individual
battles within a queue, not the results of the queues themselves. We
believe this makes our work more readily applicable to deck-building
games which are not auto-battlers, and we consider balancing
eight-player queues out of scope for this exploratory essay.  Going
forward, we will refer to a multiplayer queue as a metagame, and a
two-player match as a game, but note that this terminology varies
among auto battlers.
% TODO: move this, and rephrase here that it's "out of scope", not
% "future work", but mention that the tencent people did this
% Future work could evaluate the feasibility of modifying our approach
% to optimize the balance of entire eight-player queues (which we call
% metagames), rather than two-player matches (which we call games).

Auto battlers are a natural fit for our research, as computing the
expected payoffs of a game between two decks does not require
developing or training an AI player agent. Also, decks in auto battler
games tend to be significantly smaller than in active-play
deck-building games; both \textit{Hearthstone Battlegrounds}
\cite{hearthstone-battlegrounds} and \textit{Storybook Brawl}
\cite{storybook-brawl} use seven-card decks, whereas popular
deck-building games like \textit{Magic: the Gathering}
\cite{magic-the-gathering}, \textit{Yu-Gi-Oh!} \cite{yugioh-tcg} and
\textit{Pok\'{e}mon} \cite{pokemon-tcg} tend to use 40- to 60-card
decks.
% TODO: do i need to cite these?
The smaller deck size dramatically reduces the set of popular decks,
enabling our framework to construct and evaluate every possible deck
from a given set of cards.

% TODO: move to related works

% \citeauthor{tencent_autobattle_lineup} have shown promising results
% using neural networks to identify powerful decks in an auto battler
% metagame \cite{xumining}. Integrating their strategy with our
% optimization framework could allow the designers of larger sets to
% more efficiently balance large metagames by considering only strong
% decks rather than evaluating the entire field. In addition to
% requiring less computation, this would better approximate human play
% patterns, since human players tend to avoid building weak decks, and
% therefore evaluating a candidate deck's performance against weak
% decks matters less.
