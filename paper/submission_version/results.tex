The first experiment we ran was to determine the efficacy of our sampling method. To do this, we compared the win rates of identical decks in a complete round robin tournament (every combination of decks played) with the win rate in a tournament conducted by randomly grouping decks into groups of various sizes, and then playing every combination of the decks within those groups. See for example:

%% Table

We then compute the average error over all of the decks, and we can see a steady convergence of the two values.

% Another table

The next experiment we ran was to optimize a set of four basic cards, with no extra mechanics. 

We then ran an experiment to optimize a number of cards, each with a special mechanic. Each of these cards was assigned an intuitive value for it's attack and health, as a game designer might do. We then optimized over the stats that controlled the special mechanics of these cards. We were interested to see how well it would be possible to optimize the game with these constraints, and how impactful the special mechanics would be for the overall balance of the game. The cards used in this experiment are as follows:

\begin{tabular}{||c c c c||} 
 \hline
 Card & Attack & Health & Variable Mechanic \\ [0.5ex] 
 \hline\hline
 Explode On Death & 2 & 1 & Damage dealt to enemies on death \\ 
 \hline
 Friendly Vampire & 1 & 3 & Amount of heath granted to ally \\
 \hline
 Grow On Damage & 0 & 5 & Attack gained per hit taken \\
 \hline
 Heal On Death & 1 & 2 & Health granted to allies on death \\
 \hline
 Health Donor & 1 & 4 & Percent of healing received that is split amongst allies \\
 \hline
 Ignore First Damage & 2 & 1 & Number of attacks which result in no damage taken \\
 \hline
 Morph Attack & 0 & 3 & N/A \\
 \hline
 Pain Splitter & 2 & 2 & Percent of damage taken that is split amongst allies \\
 \hline
 Rampage & 0 & 4 & Middle Age (see section ...) \\
 \hline
 Survivalist & 2 & 2 & N/A \\
 \hline
 Threshold & 2 & 2 & Number of battles it must survive to damage all opponents on death \\
 \hline
 Time Bomb & 1 & 8 & Number of interactions before it will explode during the next battle \\ 
 \hline
\end{tabular}

