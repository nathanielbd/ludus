% shall we have a discussion section after methods and results?
% Discussion
% 	- possibility of using the "domination graph" as data,
% 	where decks are nodes and win rates are the directed
% 	edges
% 	- since we operate on discrete distributions, if you
% 	know your "ideal" distribution, you could use the 
% 	Wasserstein-1 metric as your loss function for
% 	optimization
% 	- there is some correlation in the win rate dictionary
% 	distribution, as cards appear alongside other cards

\subsection{Playtesting simulator}

To serve as a data source for quantitative analysis, we create
a playtesting simulator of a simplified auto battler game. 
This simulator is designed for research. It allows the user to 
create cards with new mechanics and run tournaments between
arbitrary decks. 

Cards have nonnegative integer statistics including health points, 
attack, and statistics pertaining to a special mechanism. Such statistics
developed include the following:

\begin{itemize}
	\item Explosion damage. When the card dies, it damages
	each of the opponent's cards by this many health points.
	\item Heal amount. Prior to entering combat, the card 
	restores this many health points to its player's rightmost
	card
	\item Attack growth per hit. When the card takes damage, its
	attack increases by this many points.
	\item Explosion heal. When the card dies, it heals
	each of its player's other cards by this many health points.
	\item Heal donation percent. When the card would be healed,
	it instead restores this many percent of the health points
	to itself and donates the remaining health points to its
	player's other cards evenly.
	\item Armor points. The card starts with this many armor points.
	When the card would be damaged, it instead loses an armor point.
	When the card has zero armor points, it may be damaged normally.
	\item Damage split percent. When the card would be damaged,
	it instead receives this many percent of the damage to itself
	and splits the remaining damage evenly to its player's other cards.
	\item Middle age. Prior to entering combat, the card increases its
	attack by 1 if it has not participated in this many combat phases.
	After this many combat phases, the card decreases its attack by 1
	before entering combat.
	\item Target age. When the card dies, if it participated in this many
	combat phases, deal this many damage points to each of the opponent's
	cards. If the card participated in fewer than this many combat phases, 
	heal each of its player's other cards by health points equal to the
	number of combat phases in which the card participated.
	\item Detonation time. When the card has been healed or damaged this
	many times, the card and its opponent both die.
\end{itemize}

Other card mechanics without statistics include the ability to copy the 
attack of the opponent's cards (morphing enemies) and the ability to swap 
its health points and attack whenever its attack is greater than its health 
points (survivalist).

For the simulator, gameplay proceeds as follows:

\begin{enumerate}
	\item Each player in a tournament selects an ordered list
	of cards to be their deck.
	\item Prior to combat, the deck is arranged left-to-right.
	\item Each combat phase pits the leftmost surviving card
	of each player against each other. The cards take damage by 
	reducing their health points by their opponent card's attack.
	Cards with zero or fewer health points are dead. Surviving cards
	are arranged to be the rightmost card of their respective decks.
	\item Repeat combat until one or fewer players has a surviving card
	or the maximum number of combat phases allowed has been reached.
	If both or neither players a surviving card, the game is a tie.
	Otherwise, the player with a surviving card wins.
	\item Proceed with the tournament in a round-robin fashion. Each 
	player plays a game with each other player once.
\end{enumerate}

\subsection{Metrics}

Quantitative measures of metagame health are needed to guide 
card statistic optimization. We present metrics built upon the
output data of the playtesting simulator described previously.

The playtesting simulation outputs a data vector $v$.
Each entry $v_i \in [-1, 1]$ represents the average payoff of
the deck at index $i$ during the last round of playtesting in
the optimization process. A payoff of 1 indicates a win. A
payoff of 0 means a tie, and a loss results in a payoff of -1.
The average of these payoffs over the course of the simulated
tournament for a deck make up the entries of $v$.
% additional sentence to clarify that the vector is averaged

% explicitly say that n is the length of w in its own sentence
% use d instead of n as it is the number of decks
% avoid using the word dictionary -- just stick to vector only
We then transform $v$ into another vector of length $n$ which maps a card to the
average win rate of the decks in which the card appears. This
vector $w$ has entries
$w_j \in [0, 1]$, the average win rate of the decks in which
the card at index $j$ appears. We present three metrics that 
use this distribution of win rates among the cards.

% what to add?
% whether to minimize or maximize
% some intuition
% run Jack's `plot.py` script

% \subsubsection{Per-card payoff}

% \begin{equation}
% 	\mathrm{PCP} = \frac{1}{n} \sum_j \left|\mathrm{payoff}(w_j)\right|^2
% \end{equation}

\subsubsection{Standard deviation metric}

\begin{equation}
	\mathrm{SDM} = \sqrt{\frac{1}{n} \sum_j \left|w_j - \left(\frac{1}{n}\sum_j w_j\right)\right|^2}
\end{equation}

Standard deviation of the win rates is a metric we may seek to minimize. Doing so may avoid decks with
very high or low win rates compared to others.

\subsubsection{Entropy metric}

\begin{equation}
	\mathrm{EM} = -\sum_j \frac{w_j}{\sum_j w_j} \log\left(\frac{w_j}{\sum_j w_j}\right)
\end{equation}

Entropy of the win rates is a metric we may seek to maximize. Doing so may avoid decks with very high or
low win rates compared to others.

\subsection{Optimization}

% We use the L-BFGS-B optimization algorithm included in SciPy. This will attempt to find the optimal values of 
% parameters, which may include the health points, attack points, and stats pertaining to a 
% card's special ability.

We use the PyGAD python library for genetic optimization algorithms. This will attempt to find the optimal values of
parameters, which may include health points, attack, and statistics pertaining to a special mechanism, that optimizes
the chosen metric.

\subsection{Qualitative analysis}

% describe Jack's plotting method

Another way of understanding the landscape of the metagame is through more qualitative methods. We look at two different ways of plotting data about the game results which give insights into the health and stability of the metagame. 

The first plot is a sweep over two variables, typically of the same card, although not necessarily. This gives a visual representation of the metagame throughout various configurations, and can lead to some interesting insights about the relationship between cards. 



\subsection{Experiments}

\subsubsection{Group versus round-robin tournament}

% \subsubsection{Metrics comparison}

\subsubsection{Optimizing cards without special mechanisms}

\subsubsection{Optimizing only statistics for special mechanisms}

\subsubsection{Optimizing after a set rotation}

\subsubsection{Optimizing health points, attack, and statistics for special mechanisms}

% \subsubsection{Optimizing an exhaustive round-robin tournament}