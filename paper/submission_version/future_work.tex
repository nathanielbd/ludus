% more metrics, other optimization methods

\subsection{Other auto battler features}

% draft phase of auto battlers

As mentioned in the introduction, the broader context of auto battlers is that
players in a queue iteratively build their deck between battles by purchasing
cards from an array of options. While our experiments were concerned with decks of size 3, 
the ability of our framework to quantify, qualify, and optimize balance for decks with 
more or fewer cards remains to be assessed. Additionally, we have previously only considered
the case where any card may be replaced with any other card to create a new deck. However, 
the purchasing value of cards may be different and future work should consider comparing decks
which may be built after the same number of battles and purchasing phases.

% cards with multiple mechanics

In many auto battlers, cards have multiple mechanics which may or may not have special parameters.
Our work only considers cards with a single mechanic, but should be extended to include such cards.

% non-determinism/random effects? is this in other games besides heartstone battlegrouds?

\subsection{Simulation data}

All metrics we present are based off of the average win rates of decks collected from the simulator. 
This does not take into account individual deck-versus-deck win rates, which are always 0 or 1 since
our simplified auto battler game is deterministic. One could construct a \textit{domination graph} with decks
as nodes and a directed edge to the deck that wins the head-to-head matchup. The size of the domination
graph problematically scales with the square of the number of decks just like the round-robin tournament,
so another sampling method would be necessary in practice. Further investigation is required to determine if perturbation of card parameters causing a degradation
in metagame health corresponds to any phase transition in a domination graph.

% We don't consider deck vs deck win rates, only average win rates

% mention our original pareto idea?

\subsection{Alternative optimization methods}

Our choice of a genetic algorithm for optimization is somewhat arbitrary. Genetic algorithms can conveniently 
handle the familiar integer values of card parameters, unlike other optimization methods. Gradient- and Hessian-based
methods in particular are difficult to apply to this problem due to its discrete nature and our simulator not being
differentiable. Had they been applicable, they would have solved an unmet need of qualifing the stability of optimization
results with respect to the variable parameters.

% dumb idea as the simulation is not differentiable
% One benefit of using a different optimization method is
% providing sensitivity analysis; some algorithms, like the Broyden--Fletcher--Goldfarb--Shanno (BFGS) algorithm, 
% approximate the Hessian of the optimization function, which could be useful in finding the sensitivity of the
% metagame health with respect to card parameters. Further consideration is needed to develop a version of these
% types of optimization 