We can see that the {\sc Ludus} framework yielded some very useful results from a game design perspective. Given a small, yet representative, set of cards, our methods were able to find a much more balanced configuration, that would likely be a far more enjoyable game to play.

We also implemented an approximation method that considerably reduces the compute time necessary to run optimizations, especially for large card sets. This method randomly chunks the tournament into smaller groups, which only compete within themselves. We tested this method, and determined that under most circumstances, the approximations it produces are close enough to the result of the larger tournament to produce satisfactory optimization results. 

We applied the {\sc Ludus} framework to some standard problems faced by game designers, and found promising results. Notably, the algorithm was successful in optimizing a second set of cards after fixing a first, previously optimized, set.

Beyond the immediate results of this paper, we have contributed an auto battler game, and its associated tooling to the literature. Given the recent nature of this genre, we believe this is an important contribution that will aid future research in the area.